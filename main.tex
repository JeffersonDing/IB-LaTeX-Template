%%%%%%%%%%%%%%%%%%%%%%%%%%%%%%%%%%%%%%%%%
% Jefferson Ding's IB LaTeX Template
% Modified from OIST Doctoral Thesis
% LaTeX Template
% Version 0.1 (2021/11)
%
% Author:
% Jefferson Ding
% Original author:
% Jeremie Gillet
%
%%%%%%%%%%%%%%%%%%%%%%%%%%%%%%%%%%%%%%%%%

%-------------------------------------------------------------------------------
%	REQUIRED PACKAGES AND  CONFIGURATIONS
%-------------------------------------------------------------------------------

\documentclass[temporary]{ib_template} % Temporary version for thesis revision and examination
%\documentclass[final]{ib_template} % Final version for thesis submission

% The documentclass ib_template includes the following packages: geometry, caption, xkeyval

\usepackage[english]{babel} % The document is in English
\usepackage[utf8]{inputenc} % UTF8 encoding
\usepackage[T1]{fontenc} % Font encoding

\usepackage{graphicx} % For including images
\graphicspath{{./Images/}} % Specifies the directory where pictures are stored


\usepackage{setspace} % For using single or double spacing
\usepackage{longtable} % tables that can span several pages
\usepackage{pdfpages} % To include a pdf files of your published papers as an appendix
\usepackage{fancyhdr} % For the headers
\usepackage{hyperref} % Adds clickable links at references

%----------------------------------------------------------------------------------------
%	ADD YOUR PACKAGES (be careful of package interaction)
%----------------------------------------------------------------------------------------

\usepackage{amsthm,amsmath,amssymb,amsfonts,bbm}% Math symbols

%----------------------------------------------------------------------------------------
%	ADD YOUR DEFINITIONS AND COMMANDS
%----------------------------------------------------------------------------------------

% Example of New Commands
\newcommand{\bea}{\begin{eqnarray}} % Shortcut for equation arrays
\newcommand{\eea}{\end{eqnarray}}
\newcommand{\e}[1]{\times 10^{#1}}  % Powers of 10 notation

% Example of Defining a theorem box for Criteria 
\newtheorem{critere}{Criterion}
\newcommand{\crit}[2]{
	\begin{center}
		\fbox{ \begin{minipage}[c]{0.9 \textwidth}
				\begin{critere}
					\textbf{\textup{ #1}} --- #2
				\end{critere}
	\end{minipage}  } \end{center}
}

%----------------------------------------------------------------------------------------
%	PICK YOUR BIBLIOGRAPHY STYLE
%----------------------------------------------------------------------------------------

\usepackage[square, numbers, sort&compress]{natbib} % for bibliography - Square brackets, citing references with numbers, citations sorted by appearance in the text and compressed (as in [4-7])
%\usepackage[longnamesfirst,round]{natbib} % Natural Sciences bibliography

\bibliographystyle{Sources/apa} % You may use a different style adapted to your field
% \bibliographystyle{apa} % You may use a different style adapted to your field

%-------------------------------------------------------------------------------
%	TITLE PAGE
%-------------------------------------------------------------------------------

\begin{document}
\pagestyle{empty} % No page numbers

\putee{
	ibtitle=\LaTeX\ IB EE Template, % Title of the thesis
	rq = {This is an example research question?},
	name = John Doe,
	subject = Subject,
	submissiondate={March 2018},  % Submission date "Month, year"
	wordcount={1000}
}

% \putia{
% 	ibtitle=\LaTeX\ IB IA Template, % Title of the thesis
% 	subject = Subject,
% 	name = John Doe,
% 	supervisor = {Jane Doe},
% 	submissiondate={May 2020},  % Submission date "Month, year"
% 	wordcount={1000},
% 	ibnum={000000}
% }


%-------------------------------------------------------------------------------
%	PREAMBLE PAGES (delete unnecessary pages)
%-------------------------------------------------------------------------------

\startpreamble

% \input{Preamble/declaration}

\unnumberedsection{Abstract} 
\section*{Abstract} 
\subsection*{\ibtitle}

 The abstract should fit within a page. 
% \unnumberedsection{Acknowledgment} 
\section*{Acknowledgment} 

Please refer to \url{https://groups.oist.jp/grad/academic-program-policies} for specifications.
% \input{Preamble/abbreviations}
% \input{Preamble/glossary}
% \input{Preamble/nomenclature}
% \input{Preamble/dedication}

%-------------------------------------------------------------------------------
%	LIST OF CONTENTS/FIGURES/TABLES
%-------------------------------------------------------------------------------
\unnumberedsection{Contents}
\tableofcontents % Write out the Table of Contents
% \unnumberedsection{List of Figures}
% \listoffigures % Write out the List of Figures
% \unnumberedsection{List of Tables}
% \listoftables % Write out the List of Tables

%-------------------------------------------------------------------------------
%	THESIS MAIN TEXT
%-------------------------------------------------------------------------------

\addtocontents{toc}{\vspace{2em}} % Add a gap in the Contents, for aesthetics

\unnumberedsection{Introduction} % Title of the unnumbered chapter
\input{MainText/introduction} % Introduction (unnumbered)

\numberedsection % Regular chapters following
\section{How to Use This Template} \label{ch-1}

 % Input your chapters here
% \section{Lorem Ipsum: Two} \label{ch-template}

Lorem ipsum dolor sit amet consectetur adipisicing elit. Maxime mollitia,
molestiae quas vel sint commodi repudiandae consequuntur voluptatum laborum\citep{Fil09}
numquam blanditiis harum quisquam eius sed odit fugiat iusto fuga praesentium
optio, eaque rerum! Provident similique accusantium nemo autem.Veritatis
obcaecati tenetur iure eius earum ut molestias architecto voluptate aliquam
nihil, eveniet aliquid culpa officia aut! Impedit sit sunt quaerat, odit,
tenetur error, harum nesciunt ipsum debitis quas aliquid. Reprehenderit,
quia.
% \input{MainText/section3}
%\input{MainText/chapter4}
%\input{MainText/chapter5}

\unnumberedsection{Conclusion} % Title of the unnumbered chapter
\input{MainText/conclusion} % Conclusion (unnumbered)

%-------------------------------------------------------------------------------
%	BIBLIOGRAPHY
%-------------------------------------------------------------------------------

\addtocontents{toc}{\vspace{1em}} % Add a gap in the Contents, for aesthetics
\unnumberedsection{Bibliography} % Title of the unnumbered chapter
\bibliography{Sources/bibliography} % The references information are stored in the file named "Thesis_bibliography.bib"

%-------------------------------------------------------------------------------
%	APPENDICES (optional)
%-------------------------------------------------------------------------------

\addtocontents{toc}{\vspace{1em}} % Add a gap in the Contents, for aesthetics
\appendix
\numberedsection % Regular chapters following
\input{MainText/appendixA}
%\input{MainText/appendixB}
%\input{MainText/appendixC}

\end{document}
