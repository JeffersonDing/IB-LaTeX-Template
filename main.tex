%%%%%%%%%%%%%%%%%%%%%%%%%%%%%%%%%%%%%%%%%
% Jefferson Ding's IB LaTeX Template
% Modified from OIST Doctoral Thesis
% LaTeX Template
% Version 0.1 (2021/11)
%
% Author:
% Jefferson Ding
% Original author:
% Jeremie Gillet
%
%%%%%%%%%%%%%%%%%%%%%%%%%%%%%%%%%%%%%%%%%

%-------------------------------------------------------------------------------
%	REQUIRED PACKAGES AND  CONFIGURATIONS
%-------------------------------------------------------------------------------

\documentclass[temporary]{ib_template} % Temporary version for thesis revision and examination
%\documentclass[final]{ib_template} % Final version for thesis submission

% The documentclass ib_template includes the following packages: geometry, caption, xkeyval

\usepackage[english]{babel} % The document is in English
\usepackage[utf8]{inputenc} % UTF8 encoding
\usepackage[T1]{fontenc} % Font encoding

\usepackage{graphicx} % For including images
\graphicspath{{./Images/}} % Specifies the directory where pictures are stored


\usepackage{setspace} % For using single or double spacing
\usepackage{longtable} % tables that can span several pages
\usepackage{pdfpages} % To include a pdf files of your published papers as an appendix
\usepackage{fancyhdr} % For the headers
\usepackage{hyperref} % Adds clickable links at references

%----------------------------------------------------------------------------------------
%	ADD YOUR PACKAGES (be careful of package interaction)
%----------------------------------------------------------------------------------------

\usepackage{amsthm,amsmath,amssymb,amsfonts,bbm}% Math symbols

%----------------------------------------------------------------------------------------
%	ADD YOUR DEFINITIONS AND COMMANDS
%----------------------------------------------------------------------------------------

% Example of New Commands
\newcommand{\bea}{\begin{eqnarray}} % Shortcut for equation arrays
\newcommand{\eea}{\end{eqnarray}}
\newcommand{\e}[1]{\times 10^{#1}}  % Powers of 10 notation

% Example of Defining a theorem box for Criteria 
\newtheorem{critere}{Criterion}
\newcommand{\crit}[2]{
	\begin{center}
		\fbox{ \begin{minipage}[c]{0.9 \textwidth}
				\begin{critere}
					\textbf{\textup{ #1}} --- #2
				\end{critere}
	\end{minipage}  } \end{center}
}

%----------------------------------------------------------------------------------------
%	PICK YOUR BIBLIOGRAPHY STYLE
%----------------------------------------------------------------------------------------

\usepackage[square, numbers, sort&compress]{natbib} % for bibliography - Square brackets, citing references with numbers, citations sorted by appearance in the text and compressed (as in [4-7])
%\usepackage[longnamesfirst,round]{natbib} % Natural Sciences bibliography

\bibliographystyle{Sources/apa} % You may use a different style adapted to your field
% \bibliographystyle{apa} % You may use a different style adapted to your field

%-------------------------------------------------------------------------------
%	TITLE PAGE
%-------------------------------------------------------------------------------

\begin{document}
\pagestyle{empty} % No page numbers

\putee{
	ibtitle=\LaTeX\ IB EE Template, % Title of the thesis
	rq = {This is an example research question?},
	name = John Doe,
	subject = Subject,
	submissiondate={March 2018},  % Submission date "Month, year"
	wordcount={1000}
}

% \putia{
% 	ibtitle=\LaTeX\ IB IA Template, % Title of the thesis
% 	subject = Subject,
% 	name = John Doe,
% 	supervisor = {Jane Doe},
% 	submissiondate={May 2020},  % Submission date "Month, year"
% 	wordcount={1000},
% 	ibnum={000000}
% }


%-------------------------------------------------------------------------------
%	PREAMBLE PAGES (delete unnecessary pages)
%-------------------------------------------------------------------------------

\startpreamble

% \unnumberedsection{Declaration of Original and Sole Authorship} 
\section*{Declaration of Original and Sole Authorship} 

I, \name, declare that this thesis entitled \emph{\ibtitle} and the data presented in it are original and my own work. 


I confirm that:
\begin{itemize}
\item No part of this work has previously been submitted for a degree at this or any other university.
\item References to the work of others have been clearly acknowledged. Quotations from the work of others have been clearly indicated, and attributed to them.
\item In cases where others have contributed to part of this work, such contribution has been clearly acknowledged and distinguished from my own work.
\item None of this work has been previously published elsewhere, with the exception of the following: (provide list of publications or presentations, or delete this part).  (If the work of any co-authors appears in this thesis, authorization such as a release or signed waiver from all affected co-authors must be obtained prior to publishing the thesis.  If so, attach copies of this authorization to your initial and final submitted versions, as a separate document for retention by the Graduate School, and indicate on this page that such authorization has been obtained).  
\end{itemize}

Date:  \submissiondate

Signature: \textbf{You may include here an image with a scan of your signature.}







\unnumberedsection{Abstract} 
\section*{Abstract} 
\subsection*{\ibtitle}

 The abstract should fit within a page. 
% \unnumberedsection{Acknowledgment} 
\section*{Acknowledgment} 

Please refer to \url{https://groups.oist.jp/grad/academic-program-policies} for specifications.
% \unnumberedsection{Abbreviations} 
\section*{Abbreviations} 

Please refer to \url{https://groups.oist.jp/grad/academic-program-policies} for specifications.

Here is an example.

\begin{longtable}{rl}
PPT & positive partial transpose\\
SRPT & Schr\"odinger-Robertson partial transpose
\end{longtable}
% \unnumberedsection{Glossary} 
\section*{Glossary} 

Please refer to \url{https://groups.oist.jp/grad/academic-program-policies} for specifications.

Here is an example:

% Break up this table into several ones if it takes up more than one page
\begin{center}
\begin{longtable}{r p{0.58 \textwidth}}
Dipole Blockade & Phenomenon in which the simultaneous excitation of two atoms is inhibited by their dipolar interaction. \\
Cavity Induced Transparency & Phenomenon in which a cavity containing two atoms excited with light at a frequency halfway between the atomic frequencies contains the number of photons an empty cavity would contain.  \\ 
\end{longtable}
\end{center}

% \unnumberedsection{Nomenclature} 
\section*{Nomenclature} 

Please refer to \url{https://groups.oist.jp/grad/academic-program-policies} for specifications.

Here is an example:

% Break up this table into several ones if it takes up more than one page
\begin{longtable}{rl}
$c$ & Speed of light ($2.997\ 924\ 58 \e{8}\ \mbox{ms}^{-1}$) \\
$\hbar$ & Planck constant ($1.054\ 572\ 66\e{-34}\ \mbox{Js}$) \\
$k_B$ & Boltzmann constant  ($1.380\ 658\e{-23}\ \mbox{JK}^{-1} $) \\
$Z_0$ &Impedance of free space  ($376.730\ 313\ 461\ \Omega) $ \\
$\mu_0$ &Permeability of free-space ($4\pi\e{-7}\ \mbox{Hm}^{-1}$) \\
\end{longtable}
% \cleardoublepage
\thispagestyle{empty} % Page style needs to be empty for this page

\vspace*{8cm} 

\hfill
\begin{parbox}{0.6\textwidth}{
\begin{flushright}

If desired, an optional and short dedication may be included here.

\end{flushright}}
\end{parbox}




%-------------------------------------------------------------------------------
%	LIST OF CONTENTS/FIGURES/TABLES
%-------------------------------------------------------------------------------
\unnumberedsection{Contents}
\tableofcontents % Write out the Table of Contents
% \unnumberedsection{List of Figures}
% \listoffigures % Write out the List of Figures
% \unnumberedsection{List of Tables}
% \listoftables % Write out the List of Tables

%-------------------------------------------------------------------------------
%	THESIS MAIN TEXT
%-------------------------------------------------------------------------------

\addtocontents{toc}{\vspace{2em}} % Add a gap in the Contents, for aesthetics

\unnumberedsection{Introduction} % Title of the unnumbered chapter
\section*{Introduction}  % Name of the unnumbered section

This is the introduction. You might want to leave it unnumbered, as it is now. If you want to number it, treat it like any other chapter. % Introduction (unnumbered)

\numberedsection % Regular chapters following

\section{Guidelines on the preparation of theses} \label{ch-1}
Please refer to \url{https://groups.oist.jp/grad/academic-program-policies} for specifications.

Many of the formatting requirements such as page size, fonts, etc are built-in into this template. Do not modify them.

For the bibliography, we recommend using BibTeX or BibLaTeX and through the file \texttt{Preamble/Thesis\_bibliography.bib}. Citing one reference can be done like so: \cite{Lee98} and multiple references in one go like so \cite{Fil09, Muc10, Kra27}.
 % Input your chapters here
% \section{How to use the templates} \label{ch-template}

This is a practical guide into how to use this template, by explaining the role of the different folders, and an option of \verb|\documentclass{oist_thesis}|, which accepts either \verb|temporary| or \verb|final|.

\section{Folders}

The main folder contains three folders detailed here:

\begin{itemize}

\item \textbf{Images.} This folder should contain all the images that you will use in your thesis. It can contain subfolders, for example one for each chapter. To include an image from the main text, use something like \texttt{\textbackslash includegraphics\{subfolder/image.jpg\} } without worrying about the \texttt{Images} path.

\item \textbf{MainText.} This folder contains a series of \LaTeX\ files that form the main text: introduction, chapters, conclusion, appendices and published articles. The introduction and conclusion as they are now are not numbered, which creates a few difficulties with the headers of the thesis. Those are solved by including the commands \texttt{\textbackslash unnumberedsection\{\}} and \texttt{\textbackslash numberedsection} before including the files in \texttt{xxx\_Thesis.tex}. If you want the introduction and conclusion to be numbered, re-write and treat them as regular chapters.

\item \textbf{Preamble.} This folder contains a series of \LaTeX\ files with the pages that will appear before the main text. Please write (or copy and paste) your own text in those files and delete the dummy text when appropriate. The files are:
\begin{itemize}
\item \texttt{abbreviations.tex} --- List of abbreviations. If the list goes over one page, create another table.
\item \texttt{abstract.tex} --- Abstract. Follow directions in the file.
\item \texttt{acknowledgments.tex} --- Acknowledgments. Follow directions in the file.
\item \texttt{declaration.tex} --- Declaration of Original and Sole Authorship. Only modify the last item. This page needs to be signed once printed.
\item \texttt{dedication.tex} --- Dedication (optional). Should only be a very few lines.
\item \texttt{glossary.tex} --- Glossary (optional). If the list goes over one page, create another table.
\item \texttt{nomenclature.tex} --- Nomenclature (optional). If the list goes over one page, create another table.
\item \texttt{physics\_bibstyle.bst} --- Bibliography style file modified by Jeremie Gillet in 2011 to suit his thesis. Might be suitable for physics. If you want to use another custom bibliography style, include the file in this folder.
\item \texttt{Thesis\_bibliography.bib} --- BibTeX file containing your bibliography.
\end{itemize}

\end{itemize}

\section{\texttt{Thesis.tex}}

This is the main file, the only one that needs to be compiled to build the thesis. Compile once with \LaTeX, once with BibTeX and finally twice with \LaTeX\ to get all the references right. At the top of this file, you can see \verb|\documentclass[temporary]{oist_thesis}|. When you submit a temporary version to the graduate school, do not modify it. When you submit a final version, use \verb|\documentclass[final]{oist_thesis}| instead. 

Let's go through each section and comment them briefly. The last section will emphasize the differences between options \verb|\documentclass[temporary]{oist_thesis}| and \verb|\documentclass[final]{oist_thesis}|.

\subsection{PACKAGES AND OTHER DOCUMENT CONFIGURATIONS}

This section contains the minimum number of packages and definitions to compile the thesis. No line should be removed or modified.

\subsection{ADD YOUR CUSTOM VALUES, COMMANDS AND PACKAGES}

This section should not be modified directly. Instead, your packages and definitions should be included in  \texttt{Preamble/mydefinitions.tex}.

\subsection{TITLE PAGE}

Creates the title page. Do not modify.

\subsection{PREAMBLE PAGES}

Structures the style (header) for the preamble pages and builds them. Do not modify, except for deleting the optional preambles you might not want to include.

\subsection{LIST OF CONTENTS/FIGURES/TABLES}

Creates the different lists. Do not modify.

\subsection{THESIS MAIN TEXT}

Structures the style for the main text chapters and builds them. 

The command \texttt{\textbackslash numberedsection} is only relevant for a transition between unnumbered sections and numbered sections, it does not need to be included between each chapter. 

\subsection{BIBLIOGRAPHY}

Builds the bibliography. The style of the bibliography can be defined in \texttt{Preamble/mydefinitions.tex}.

\subsection{APPENDICES}

Structures the style for the appendices and builds them. The appendices are numbered with letters but are structured like regular chapters.

\subsection{PUBLISHED ARTICLES}

This last section add the PDF files of your previously published articles (or about to be published) to the thesis. You should only include PDF files provided by the publishing journal. This is strictly for the examiners' convenience in the temporary bound thesis, as for copyright reasons these files may not be published in the final version of the thesis.

\subsection{Differences between a temporary version and final version}

There are two main differences between \verb|\documentclass[temporary]{oist_thesis}| and \verb|\documentclass[final]{oist_thesis}|. 

The first difference is that the final version (\verb|\documentclass[final]{oist_thesis}|) does not contain the published articles for copyright reasons. 

The second difference is in the document style: page size, header and line spacing are different This might create small issues, such as page breaking with large tables, images or captions, when compiling the same content.




% \section{Figures, tables and images} \label{chap-3}

\section{Figures}

\begin{figure}
\center
\includegraphics[width=0.3\textwidth]{chap3/emblem.jpg} 
\caption[Short caption for List of Figures]{{\bfseries Short caption (if wanted).} Full caption with all the details here.}
\label{fig-example}
\end{figure}

\begin{figure}
\center
\includegraphics[width=0.3\textwidth]{chap3/symbol.jpg} 
\caption*{This secret image won't be numbered and won't appear in the List of Figures because of the *}
\end{figure}

Refer to figure like this: Figure~\ref{fig-example} or this (Fig.~\ref{fig-example}). If you want to include a list of figure, you can use a short version of the caption as shown in Figure~\ref{fig-example}.


\section{Tables}

\begin{table} 
\center
\caption{Short heading for the List of Tables.}
\begin{tabular}{c|c}
Parameter & Value \\ \hline \hline
$\Delta$ & 0, 150 \\
${\alpha}$ & 85 \\
${\epsilon}$ & 6 \\
${\kappa}$ & 6.8 \\
${\gamma}$ & 0.2
\end{tabular}
\label{tab-values}
\caption*{Full caption with all the details here.}
\end{table}

\begin{table} \center
\begin{tabular}{c|c}
Parameter & Value \\ \hline \hline
$\Delta$ & 0, 1500 \\
${\alpha}$ & 850 \\
${\epsilon}$ & 60 \\
${\kappa}$ & 68 \\
${\gamma}$ & 2
\end{tabular}
\caption*{This secret table won't be numbered and won't appear in the List of Figures because of the * }
\end{table}


Refer to tables this this: Table~\ref{tab-values}.
%\input{MainText/chapter4}
%\input{MainText/chapter5}

\unnumberedsection{Conclusion} % Title of the unnumbered chapter
\section*{Conclusion}  % Name of the unnumbered section

This is the conclusion. You might want to leave it unnumbered, as it is now. If you want to number it, treat it like any other chapter. % Conclusion (unnumbered)

%-------------------------------------------------------------------------------
%	BIBLIOGRAPHY
%-------------------------------------------------------------------------------

\addtocontents{toc}{\vspace{1em}} % Add a gap in the Contents, for aesthetics
\unnumberedsection{Bibliography} % Title of the unnumbered chapter
\bibliography{Sources/bibliography} % The references information are stored in the file named "Thesis_bibliography.bib"

%-------------------------------------------------------------------------------
%	APPENDICES (optional)
%-------------------------------------------------------------------------------

\addtocontents{toc}{\vspace{1em}} % Add a gap in the Contents, for aesthetics
\appendix
\numberedsection % Regular chapters following


\section{Appendices} \label{appA}

Please refer to \url{https://groups.oist.jp/grad/academic-program-policies} for specifications.

%\input{MainText/appendixB}
%\input{MainText/appendixC}

\end{document}
